\chapter{Source Code Example}
%\label{chapter:title}

\emph{Adding source code to your report/thesis is supported with the package {\normalfont\texttt{listings}}. An example can be found below. Files can be added using {\normalfont\texttt{\textbackslash lstinputlisting[language=<language>]\{<filename>\}}}.}

\begin{lstlisting}[language=Python]
"""
ISA Calculator: import the function, specify the height and it will return a
list in the following format: [Temperature,Density,Pressure,Speed of Sound].
Note that there is no check to see if the maximum altitude is reached.
"""

import math
g0 = 9.80665
R = 287.0
layer1 = [0, 288.15, 101325.0]
alt = [0,11000,20000,32000,47000,51000,71000,86000]
a = [-.0065,0,.0010,.0028,0,-.0028,-.0020]

def atmosphere(h):
    for i in range(0,len(alt)-1):
        if h >= alt[i]:
            layer0 = layer1[:]
            layer1[0] = min(h,alt[i+1])
            if a[i] != 0:
                layer1[1] = layer0[1] + a[i]*(layer1[0]-layer0[0])
                layer1[2] = layer0[2] * (layer1[1]/layer0[1])**(-g0/(a[i]*R))
            else:
                layer1[2] = layer0[2]*math.exp((-g0/(R*layer1[1]))*(layer1[0]-layer0[0]))
    return [layer1[1],layer1[2]/(R*layer1[1]),layer1[2],math.sqrt(1.4*R*layer1[1])]
\end{lstlisting}

\section{A deep dive into the compiler}
\definecolor{light-green}{rgb}{0.5,1,0.5}
\definecolor{light-blue}{rgb}{0.5,0.5,1}
\definecolor{light-red}{rgb}{1,0.5,0.5}
\definecolor{light-yellow}{rgb}{1,1,0.5}
\definecolor{light-grey}{gray}{0.5}

\subsection{Comparison}
\url{https://dpu.dev/#g:!((g:!((g:!((g:!((h:codeEditor,i:(j:1,lang:___c,source:'int+leq(float+a,+float+b)+%7B%0A++++return+a+%3C%3D+b%3B%0A%7D%0A'),l:'5',n:'0',o:'C+source+%231',t:'0')),k:37.5,l:'4',m:50,n:'0',o:'',s:0,t:'0'),(g:!((h:codeEditor,i:(j:2,lang:___c,source:'int+leq(float+aa,+float+bb)+%7B%0A++++int+a+%3D*(int*)%26aa%3B%0A++++int+b+%3D*(int*)%26bb%3B%0A++++if+(a+%3C+0+%26%26+b+%3C+0)%0A++++++++return+b+%3C%3D+a%3B%0A++++else%0A++++++++return+a+%3C%3D+b%3B%0A%7D%0A'),l:'5',n:'0',o:'C+source+%232',t:'0')),header:(),l:'4',m:50,n:'0',o:'',s:0,t:'0')),k:47.25848563968668,l:'3',n:'0',o:'',t:'0'),(g:!((g:!((h:compiler,i:(compiler:clang-2021-3,filters:(b:'0',binary:'1',commentOnly:'0',demangle:'0',directives:'0',execute:'1',intel:'0',libraryCode:'1',trim:'1'),lang:___c,libs:!(),options:'',source:1),l:'5',n:'0',o:'clang+12+for+DPU+(rel+2021.3.0)+(Editor+%231,+Compiler+%231)+C',t:'0')),header:(),k:62.5,l:'4',m:50,n:'0',o:'',s:0,t:'0'),(g:!((h:compiler,i:(compiler:clang-2021-3,filters:(b:'0',binary:'1',commentOnly:'0',demangle:'0',directives:'0',execute:'1',intel:'0',libraryCode:'1',trim:'1'),lang:___c,libs:!(),options:'',source:2),l:'5',n:'0',o:'clang+12+for+DPU+(rel+2021.3.0)+(Editor+%232,+Compiler+%232)+C',t:'0')),header:(),l:'4',m:50,n:'0',o:'',s:0,t:'0')),k:52.74151436031331,l:'3',n:'0',o:'',t:'0')),l:'2',m:100,n:'0',o:'',t:'0')),version:4}

\subsection{Operands auto-promotion}

In section \ref{subsection:Limitations}, we mentioned the limitations of the compiler when it comes to optimizing for our target architecture. We're now going to expose one such occurrence and how to deal with it.

\begin{sloppypar}
\url{https://dpu.dev/#z:OYLghAFBqd5TKALEBjA9gEwKYFFMCWALugE4A0BIEAViAIzkA2AhgHagD63q5AzugCupVNhAByAKQAmAMwE2qJoJwBqSbIDCfIoTZEAdEg25JABgCC5iwqIA2ACyciq7IKUFMEW/TvOAVKos5Ko%2BfkSBAEYhgray0s6qfAQAXtgAlOoA7ABC1qoFofqOiSyoqIIAtoKsRNgaeZaFqgBmZBCx%2BvGJBBoAImYNoRraqfWyOcPSedOZkrn5zYVhiYQtLeqyfUGSAKw5vbvbALSqkXsHe30Ni0tB5VU1LHXq0/2qaxuBnzdNhfPXP4FUjYIjCNj3CrVWrjRpWLKAizidLMCS7cTkNgSMwY9ASTRJIQibCvWT0DFEbHIlEAaxAuzMqPEDgxlXpjKx4hx5Dx4gxfBAjMpXOR5DgsBQGBw%2BGIZEo1Do0mY7C4PH4RNEEhk8kUyjUIx0ekMxlkpks1lsJRcbg8XhWESCIXtURicQSLmSaTmCyBRXsThcZShTzqvwszTapA6bp6/UGE2GWk9sKmMxy3rhdwKADd0LUCEwSfaPgR1pttiwLocTmcq1cw1nIY8Ya88lsS2XvqWWg3/gjbqoQWDSBCg83nrDrADrKKmGiMZzubyCQJhKJSdIKVT0iikNgWDhSNQUXPmaz2ZiqTyJPzBeRhTid%2BQ6QymbIF1feVuRSjxQh4BAkroJUAAOBbYBQVAQBgoHgUeSgqsc0hmMhxyyOQLQFnUpAChAkRXpECgsKQACeEjkuQMGVNg%2BgAPJsEwZEiuQOCVCqhZXoQIKoEQBDZtgArMdgAAe2AVHU5EYrY2CntyTAEJEpDESRmhYGI4gUUQpAEGyGmziqIDcJwvDyZEAqQCi6Agbx6BsIJ/IamI9AnvOl7MbyCEcKo9BmK0ZCqH0AAKACqqgQCCTCqMhyEGLIBhmJkEAyiQpCkowqiqbBhapdqzkZd%2Bj60hep4suQbKvouuI3vwd4PqKf6AWgwFgdl8rQc1cFoKwHBIShZhoRhWEQbh%2BHMYRbDKZJlHAdRdEMUx3Ksex6mLQQ3G8fxgnciJYmCBJemUPoMlXqZSmkapOBTVpOmSSeBlGSZCnmRAlnWQQtn2eqa5iNILniOiblLhInnAN5vmRgFIVhRFUV9bF8WJcl/nakqGUddlG6ZJoBXUuQe4HhBx5MqV5Ucp%2B1UCkK25Fa%2Bp7voDVV8ve1NMpuDPXkzdVPvxOHvViDhAA}
\end{sloppypar}

The code:
\lstset{
   showlines=true
}
\begin{lstlisting}[language=C]
#include <stdint.h>

\end{lstlisting}
\vspace{-\baselineskip}
\begin{lstlisting}[language=C, backgroundcolor=\color{light-green}]
int64_t euclid(int16_t* a, int16_t* b, uint32_t size) {
    int64_t accumulate;
    for(uint32_t i=0; i<size; i++) {
        int16_t diff = a[i] - b[i];
        accumulate += diff * diff;
    }
    return accumulate;
}
\end{lstlisting}

The compiled:
\begin{lstlisting}[language={[x86masm]Assembler}]
    euclid:                                 // @euclid
    sd r22, 24, d22
    add r22, r22, 32
    sd r22, -16, d14
    sd r22, -24, d16
    sd r22, -32, d18
    move r14, r2
    move r15, r1
    move r16, r0
    jeq r14, 0, .LBB0_2
.LBB0_1:                                // =>This Inner Loop Header: Depth=1
    lhs r0, r16, 0
    lhs r1, r15, 0
    sub r0, r0, r1
    extsh r0, r0
    move r1, r0
    call r23, __mulsi3
    move.u d0, r0
    add r19, r19, r1
    addc r18, r18, r0
    add r15, r15, 2
    add r16, r16, 2
    add r14, r14, -1, nz, .LBB0_1
.LBB0_2:
    movd d0, d18
    ld d18, r22, -32
    ld d16, r22, -24
    ld d14, r22, -16
    ld d22, r22, -8
    jump r23
\end{lstlisting}